\section{Perintah pada Unix}
		\subsection{Definisi}
		\hspace*{1cm}Perintah pada UNIX merupakan perintah yang dijalankan pada sistem operasi UNIX, yang diberikan user untuk melakukan perintah yang diinginkan baik berupa perintah/command internal, ataupun perintah eksekusi suatu file program yang biasa disebut perintah/command eksternal. Program penterjemah perintah/command yang menjembati antara user dengan sistem operasi dalam hal ini kernel yaitu shell. Shell dapat digunakkan user untuk menyusun perintah pada beberapa file untuk dieksekusi sebagai sebuah program. Shell pada UNIX tidak hanya menyediakan 1 atau 2 shell saja, namun dilengkapi oleh banyak shell dengan kumpulan perintah yang sangat banyak, sehingga user dapat memilih shell mana yang lebih mudah dalam membantu menyelesaikan pekerjaannya, dan dapat berpindah pindah dengan mudah dari shell satu ke shell yang lainnya.
		\begin{figure}[ht]
			\centerline{\includegraphics[width=0.5\textwidth]{figures/command.png}}
			\caption{Contoh command UNIX}
			\label{command}
			\end{figure}
		\vspace{1cm}Ini adalah contoh beberapa command UNIX pada gambar \ref{command} 